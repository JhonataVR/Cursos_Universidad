\documentclass[12pt]{article}

% Paquetes necesarios
\usepackage[utf8]{inputenc}  % Codificación de texto
\usepackage[spanish]{babel} % Idioma español
\usepackage{amsmath, amssymb} % Paquetes para matemáticas avanzadas
\usepackage{listings} % Para mostrar código fuente
\usepackage{xcolor} % Para colorear código
\usepackage{graphicx} % Para incluir imágenes
\usepackage{hyperref} % Hipervínculos
\hypersetup{
	colorlinks=true,
	linkcolor=blue,
	filecolor=magenta,
	urlcolor=cyan,
}

% Configuración del estilo del código fuente
\lstset{
	basicstyle=\ttfamily\small, % Fuente del código
	keywordstyle=\color{blue}\bfseries,
	commentstyle=\color{green!60!black},
	stringstyle=\color{red},
	numbers=left,
	numberstyle=\tiny,
	numbersep=5pt,
	frame=single,
	breaklines=true,
	captionpos=b,
}

% Título del documento
\title{Vectores en R3 con Python}
\author{Jhonatan Vargas Ramos}
\date{\today}

\begin{document}
	
	% Portada
	\maketitle
	\begin{center}
		\textbf{Materia: Cálculo y Programación}\\
		\textbf{Fecha: \today}
	\end{center}
	\newpage
	
	% Índice
	\tableofcontents
	\newpage
	
	% Secciones principales
	\section{Introducción}
	Este informe esta dedicado para dar a conocer la fusión entre cálculo y Python. 
	
	\section{Cálculo}
	Para esta ocasión nos basaremos en el tema de vectores en R3	
	\subsection{Definiciones y conceptos básicos}
	
	\begin{itemize}
		\item Derivadas e integrales
		\item Límites
		\item Series y sucesiones
	\end{itemize}
	
	\subsection{Demostraciones}
	Ejemplo de demostración matemática:
	\[
	\lim_{x \to 0} \frac{\sin x}{x} = 1
	\]
	
	\subsection{Gráficos}
	Puedes incluir gráficos creados con herramientas externas o con TikZ si trabajas con LaTeX. Ejemplo de inclusión de una imagen:
	\begin{figure}[h!]
		\centering
		%\includegraphics[width=0.7\textwidth]{grafico.png}
		\caption{Ejemplo de gráfico}
	\end{figure}
	
	\section{Código de Programación}
	En esta sección se presenta el código relacionado con los temas de cálculo, su propósito y los resultados obtenidos.
	
	\subsection{Descripción del código}
	Explicación breve de qué hace el programa y cómo se relaciona con los temas de cálculo.
	
	\subsection{Código fuente}
	Ejemplo de inclusión de código:
	\begin{lstlisting}[language=Python, caption={Cálculo de derivadas con Python}]
		import sympy as sp
		
		x = sp.Symbol('x')
		f = sp.sin(x)
		derivada = sp.diff(f, x)
		print(derivada)
	\end{lstlisting}
	
	\subsection{Resultados}
	Incluye capturas de pantalla, gráficos generados por el código o resultados numéricos obtenidos.
	
	\section{Conclusión}
	Resumen de lo aprendido, los desafíos enfrentados y las posibles aplicaciones prácticas de los temas y el código desarrollado.
	
	\section{Referencias}
	Lista de libros, artículos, páginas web o cualquier fuente utilizada para realizar el informe. Ejemplo:
	\begin{itemize}
		\item Stewart, J. (2015). \textit{Cálculo: conceptos y contextos}.
		\item Documentación oficial de Python: \url{https://docs.python.org/3/}
	\end{itemize}
	
\end{document}
